\subsection{\uppercase{Evolution and Prediction}}
In addition to providing key insights to possible variations in LDA for event detection, {\bf Nishant Yadav} has focussed specifically on exploring the recent literature for evolution and prediction of events using social media.

\subsubsection{\uppercase{Tracking and Evolution}}
A lot of research is going on to study the evolution of events over time in general. \cite{yang2014finding} studied common patterns and progression stages in event sequences like medical records, reviews of products and services and web/search logs. They build a generative model to study the common patterns in general event evolution. They classified sequences into different classes based on common stages of evolution. \cite{duygulu2004towards} in their work tracked evolution of news stories for the purpose of event summarization. The core idea of their work is based on visual cues and textual information. News channels often repeat shots of a video multiple times during news broadcase. They have exploited this observation and used visual cues to detect repeating videos in a news to keep track of an event evolving over time. Evolution of events within a topic related to an incident using online news has been studied by \cite{yang2009discovering}. Significant research efforts have been made to study evolution of events in general. However, the effects of LDA-segmented tweets-cum-events on their tracking and evolution has not been studied in the past. Moreover, we also wish to examine the differences in evolution patterns of popular vs. non-popular events.

\subsubsection{\uppercase{Event Prediction}}
Event prediction using social media has become very popular in recent years considering the diversity of available data. \cite{asur2010predicting} used Twitter to predict box-office revenues for movies. They used a model based on the number of tweets to predict popularity of a movie; for this they analyzed the tweet traffic starting from pre-release period to few weeks post release. Further, they used sentiment analysis based on text classifiers to separate positive opinions from negative ones on a movie to better predict the box-office revenue. It is observed that positive opinions encourage other people to watch the movie thus having positive effect on revenue. Their research also shows that event predictors based on tweets are better than other predictors. Parallely, research has been done to predict stock market changes using sentiment analysis of tweets. \cite{bollen2011twitter} has used two mood tracking tools, OpinionFinder which distinguishes positive mood from negative mood using tweet feeds and Google-Profile of Mood States (GPOMS) which classifies moods using tweets on 6 different dimensions. The mood timelines thus obtained are used to predict changes in Dow Jones Industrial Average (DJIA) values.
\begin{table}
	\centering
    \begin{tabular}{|c|c|c|}
    \hline

    \multicolumn{1}{|c|}{\textbf{Topic 1}} & \multicolumn{1}{|c|}{\textbf{Topic 2}} & \multicolumn{1}{|c|}{\textbf{Topic 3}} \\
    \hline
    to  & on   & xd0  \\ 
    the & it   & i    \\ 
     my & in   & the  \\ 
     it & my   & http \\
    and & im   & you  \\ 
     be & that & xbf  \\ 
    not & have & com  \\ 
    \hline
    \end{tabular}
    \caption {LDA : Topics from Twitter Data }
\end{table}

\begin{table}
	\centering
    \begin{tabular}{|c|c|c|}
    \hline
    \multicolumn{1}{|c|}{\textbf{Topic 1}} & \multicolumn{1}{|c|}{\textbf{Topic 2}} & \multicolumn{1}{|c|}{\textbf{Topic 3}} \\
    \hline
 film	& east	& trade \\ 
 last	& german	& united \\ 
 movie	& west	& states \\ 
 social	& black	& japan \\ 
 benefits	& south	& foreign \\ 
 security	& government	& japanese \\ 
 rep	& mandela	& countries \\ 
 	\hline
     \end{tabular}
     \caption {LDA : Topics from News Articles }
\end{table}
However, research till now has inherently been constraint to prediction popular events based on sentiment analysis of tweets. There has not been much work which specifically focuses on evolution and prediction of events related to unpopular topics. We wish to understand how standard prediction techniques are insufficient for this task, and come up with suitable modifications to increase their accuracy. One promising approach is to couple the analysis of time-series behavior of different topics based on the Twitter traffic with the temporal changes in their correlation with events in other topics. This could potentially result in a model that can accurately predict future patterns of the same event, as well as predict the triggering or branching out of other events as its consequence.