\section{\uppercase{Introduction}}
Humans have a curiosity to know more about their surrounding environment. This need for information is the main factor that has contributed towards the survival of human race. For instance, the news of the outbreak of an epidemic is immediately followed by the adoption of additional health care measures by people living in the affected regions. This huge appetite for information was originally satisfied by written media like newspapers, telegraphs etc. Then, with advancements in technology came television and telephones which provided information at a much faster rate. In the present generation, the growth of Internet has completely changed the way information is shared and received. This increase in popularity and need for information sharing has led to the emergence of social networking platforms like Twitter.

Twitter has a huge user base sharing all kinds of information at a very high rate. Information shared on Twitter range from personal information like what they are eating, to local events like festival celebrations, to events having worldwide impact like forest fires. Since the users of Twitter are spread all over the world, and people usually Tweet about events almost instantaneously, it can be considered as a large media company having its reporters spread all over the world reporting events 24*7. This fact makes the study of Twitter data very important in order to model the evolution of important events through tweets. However, along with the diversity and richness of information pervading the Twitter ecosystem comes an equally huge amount of uninteresting, insignificant and noisy information, such as updates about daily chores of a user. Mining useful information from this exploding space of tweets calls for meticulous organization and structuring of data. With this motivation, we wish to explore the problem of detection, tracking, and prediction of events using micro-blog posts.

The rest of the report is organized as follows. We begin by presenting our problem statement in a formal manner in Section 2. This is followed by a discussion on the Twitter ecosystem in Section 3. Section 4 presents a brief overview of Latent Dirichlet Allocation. In Section 5, we present a comprehensive review of the numerous techniques which have been proposed in literature for the purpose of event detection in Twitter. 

This is followed by a brief discussion of Twitter LDA in Section 6, which is an interesting variant of LDA build for inferring topic distributions on Twitter data.