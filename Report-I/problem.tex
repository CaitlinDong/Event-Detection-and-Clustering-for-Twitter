\section{\uppercase{Problem Definition}}
Our overarching aim is to detect events using microblogs in social media (such as Twitter), track and model their evolution over time, and predict the occurrences of future events. Our focus is on the following three sub-problems.

\subsection{Using LDA to detect events}
The first step towards an attempt to extract useful information from Twitter data is to detect and extract real-world events from tweets. To this end, we are exploring Topic Models. In particular, we are using Latent Dirichlet Allocation (LDA) \cite{blei2003latent}. LDA is a generative process used for infering the topics present in a text corpora and classifying the documents according to these topics. Our aim is to use LDA on Twitter posts to cluster related keywords across millions of tweets under different \emph{topics}. Topics when associated with spatial and temporal data represent events. Specifically, our focus is on the analysis of \emph{non-popular} events, which is a departure from the focus on popular events in recent literature. Event detection using topic modeling techniques like LDA are unique in the sense that unlike analysis of bursty keywords, they not biased by event popularity. Hence, they can capture non-popular events as well.

\subsection{Evolution of events}
Once an event such as bomb blast, hurricane, and presidential speech have been identified through tweets, the next step is to track the evolution of these events over time and space. We are interested in investigating how they develop within their \emph{topic}, as well as analysing how their correlation to events in other topics changes over time. We aim to build an event evolution graph to study how one event triggers other events. Here also, our focus is analysis  of the activity, traffic, and evolution pattern of the non-popular events. Another challenge is to incorporate suitable modifications to standard models employed for modeling the evolution of popular events \cite{lin2010pet} so that they can better model non-popular events in particular.

\subsection{Event prediction}
Finally, having modeled the events, our aim is to employ time-series models to predict future events based on their evolution patterns. Specifically, our focus is on prediction of future occurrences of the same event, such as predicting the onset of an annual epidemic, as well as the prediction of outset of other events as a result of the event under consideration; for example, outset of protests followed by a child abuse case. A lot of work has already been done in this aspect. Our aim is to adapt existing techniques for more accurate predictions concerning non-popular events.