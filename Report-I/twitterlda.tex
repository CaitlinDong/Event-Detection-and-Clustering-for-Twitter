
\subsubsection{\uppercase{Twitter LDA}}
An issue pertaining to the use of LDA on Twitter data is the questionable assumption of considering a tweet as a mixture of topics. Given the extremely short nature of tweets, most tweets consists of a single topic. As discussed above, some studies have tackled this problem by aggregating the tweets of a user in a single document. This method, though effective, is not guaranteed to help much because of the fact that users generally express a wide variety of different topics in their tweets which may not be related to each other. This analysis will still be good if we want to profile users; but since our aim is to mine events from the topics, the possible solutions that seems feasible will be to aggregate tweets based on time, locality and hashtags.

To this end, \cite{zhao2011comparing} have proposed an effective variant of the standard LDA, called Twitter LDA. It is based on the assumption that a tweet will contain a single topic chosen from a topic distribution of a particular user. 

\paragraph{Model Description} 
The generative model makes the following assumptions. Twitter data has T number of topics. When a user wants to write a tweet, he/she selects a topic from his/her favorite list of topics, these topics will be from the T topics. Then for the selected topic, the user selects a bag of words, one by one from the distribution of words over topics. However not all words of a tweet are closely related the topic. Many of them are just common words occurring in tweets of various different topics. So for each words user decides whether it is a background word or a topic word and then chooses the word from its respective distribution.

Formally, let $\theta_u$ denotes the topic distribution for a user $u$. Let $\phi_t$ be the distribution of words for the topic $t$ and let $\phi_B$ be the distribution of background words. $\pi$ is a Bernoulli distribution which denote the choice of a word to be a background or a topic word. $\alpha,~\beta,~\gamma$ are Dirichlet parameter used for generating respective Dirichlet distributions. The plate notation for the model is given in Fig.\ref{fig:plate}. The generative algorithm is given in Fig.\ref{fig:twitterlda-algo}.

In addition to other LDA variants, {\bf Harshil Lodhi} has particularly focussed on exploring Twitter-LDA.

\subsection{\uppercase{Evolution and Prediction}}
In addition to providing key insights to possible variations in LDA for event detection, {\bf Nishant Yadav} has focussed specifically on exploring the recent literature for evolution and prediction of events using social media.

Event prediction using social media has become very popular in recent years considering the diversity of available data. \cite{asur2010predicting} used Twitter to predict box-office revenues for movies. They used a model based on the number of tweets to predict popularity of a movie; for this they analyzed the tweet traffic starting from pre-release period to few weeks post release. Further, they used sentiment analysis based on text classifiers to separate positive opinions from negative ones on a movie to better predict the box-office revenue. It is observed that positive opinions encourage other people to watch the movie thus having positive effect on revenue. Their research also shows that event predictors based on tweets are better than other predictors. Parallely, research has been done to predict stock market changes using sentiment analysis of tweets. \cite{bollen2011twitter} has used two mood tracking tools, OpinionFinder which distinguishes positive mood from negative mood using tweet feeds and Google-Profile of Mood States (GPOMS) which classifies moods using tweets on 6 different dimensions. The mood timelines thus obtained are used to predict changes in Dow Jones Industrial Average (DJIA) values.
\begin{table}
	\centering
    \begin{tabular}{|c|c|c|}
    \hline

    \multicolumn{1}{|c|}{\textbf{Topic 1}} & \multicolumn{1}{|c|}{\textbf{Topic 2}} & \multicolumn{1}{|c|}{\textbf{Topic 3}} \\
    \hline
    to  & on   & xd0  \\ 
    the & it   & i    \\ 
     my & in   & the  \\ 
     it & my   & http \\
    and & im   & you  \\ 
     be & that & xbf  \\ 
    not & have & com  \\ 
    \hline
    \end{tabular}
    \caption {LDA : Topics from Twitter Data }
\end{table}

\begin{table}
	\centering
    \begin{tabular}{|c|c|c|}
    \hline
    \multicolumn{1}{|c|}{\textbf{Topic 1}} & \multicolumn{1}{|c|}{\textbf{Topic 2}} & \multicolumn{1}{|c|}{\textbf{Topic 3}} \\
    \hline
 film	& east	& trade \\ 
 last	& german	& united \\ 
 movie	& west	& states \\ 
 social	& black	& japan \\ 
 benefits	& south	& foreign \\ 
 security	& government	& japanese \\ 
 rep	& mandela	& countries \\ 
 	\hline
     \end{tabular}
     \caption {LDA : Topics from News Articles }
\end{table}
However, research till now has inherently been constraint to prediction popular events based on sentiment analysis of tweets. There has not been much work which specifically focuses on evolution and prediction of events related to unpopular topics. We wish to understand how standard prediction techniques are insufficient for this task, and come up with suitable modifications to increase their accuracy. One promising approach is to couple the analysis of time-series behavior of different topics based on the Twitter traffic with the temporal changes in their correlation with events in other topics. This could potentially result in a model that can accurately predict future patterns of the same event, as well as predict the triggering or branching out of other events as its consequence.