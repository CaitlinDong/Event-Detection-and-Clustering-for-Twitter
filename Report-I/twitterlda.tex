\section{\uppercase{Twitter LDA}}
An issue pertaining to the use of LDA on Twitter data is the questionable assumption of considering a tweet as a mixture of topics. Given the extremely short nature of tweets, most tweets consists of a single topic. As discussed above, some studies have tackled this problem by aggregating the tweets of a user in a single document. This method, though effective, is not guarenteed to help much because of the fact that users generally express a wide variety of different topics in their tweets which may not be related to each other. This analysis will still be good if we want to profile users; but since our aim is to mine events from the topics, the possible solutions that seems feasible will be to aggregate tweets based on time, locality and hashtags.

To this end, \cite{zhao2011comparing} have proposed an effective variant of the standard LDA, called Twitter LDA. It is based on the assumption that a tweet will contain a single topic chosen from a topic distribution of a particular user. 