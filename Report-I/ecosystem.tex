\section{\uppercase{Understanding Twitter Ecosystem}}
In this world of big data, no one can ignore the impact that Twitter has in terms of data availability and data processing. On a typical day, more than 500 million tweets are posted. This amount of data was never available before. The traditional media like newspaper, articles, magazines are very different from Twitter data. Tweets provide almost real-time information and discussions of current events. However, tweets are highly fragmented and noisy, and contain non-interesting events as well, such as personal musings of users about their day-to-day activities. Moreover, the informal, ill-framed, and unstructured nature of messages adds to the noise. All these characteristics of tweets make it difficult for the traditional systems which were based on carefully written and well-structured news articles to process Twitter data. Tweets mostly contain different spellings and misspellings for a single word. Because of the 140 character limit, most of the users refrain from the use of proper punctuation and stop-words in their tweets, resulting in grammatically, semantically and syntactically messy texts.

If we consider tweets about users' daily mundane tasks, these tweets come up on Twitter in a large volume on any given day. Intermixed with this uninteresting volume of tweets is a set of equally bursty and information-rich tweets about important events. So, to differentiate between these two classes of tweets, one cannot directly use the frequency; temporal/spatial features of these class of tweets need to be considered. Tweets consisting of daily mundane tasks are evenly distributed across the timeline while tweets about important events are concentrated to a certain part of the temporal and/or spatial dimension.

\paragraph{Events}
To define an \emph{event}, we need to state what a topic is. Quoting from \cite{zhao2011comparing}, \textquotedblleft A topic is a subject discussed in one or more documents\textquotedblright. An \emph{event} is an abstract idea which has a topic, a temporal dimension, a spatial dimension, and/or entities associated with it. For example, \textquotedblleft Death of Steve Jobs\textquotedblright~was trending at Twitter. This event has a topic \textquotedblleft death\textquotedblright~which is associated with an entity \textquotedblleft Steve Jobs\textquotedblright~and it has a temporal dimension since the burst of tweets appeared in the first week of October, the time when Steve jobs died. The temporal and spatial dimensions can be found explicitly from the tweet content and/or from the tweet's meta-data. A topic alone cannot define an event. So, to make sense from the data, one first needs to find out the different topics from the given data and then, figure out whether there is an associated entity, or a temporal/spatial dimension to it or not.