\section{\uppercase{Phase 2}}
In Phase 2, we plan to conduct more concrete experiments with different variations to topic models to come up with a model that fits our domain well. Parallely, we aim to focus on modeling the evolution of detected events. To this end, we plan to work on techniques to generate event evolution graphs, and employ time-series models to predict future events. 

Event prediction using social media has become very popular in recent years considering the diversity of available data. \cite{asur2010predicting} used Twitter to predict box-office revenues for movies. They used a model based on the number of tweets to predict popularity of a movie; for this they analyzed the tweet traffic starting from pre-release period to few weeks post release. Further, they used sentiment analysis based on text classifiers to separate positive opinions from negative ones on a movie to better predict the box-office revenue. It is observed that positive opinions encourage other people to watch the movie thus having positive effect on revenue. Their research also shows that event predictors based on tweets are better than other predictors. Parallely, research has been done to predict stock market changes using sentiment analysis of tweets. \cite{bollen2011twitter} has used two mood tracking tools, OpinionFinder which distinguishes positive mood from negative mood using tweet feeds and Google-Profile of Mood States (GPOMS) which classifies moods using tweets on 6 different dimensions. The mood timelines thus obtained are used to predict changes in Dow Jones Industrial Average (DJIA) values.

However, research till now has inherently been constraint to prediction popular events based on sentiment analysis of tweets. There has not been much work which specifically focuses on evolution and prediction of events related to unpopular topics. We wish to understand how standard prediction techniques are insufficient for this task, and come up with suitable modifications to increase their accuracy. One promising approach is to couple the analysis of time-series behavior of different topics based on the Twitter traffic with the temporal changes in their correlation with events in other topics. This could potentially result in a model that can accurately predict future patterns of the same event, as well as predict the triggering or branching out of other events as its consequence.